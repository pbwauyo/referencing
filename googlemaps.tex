\documentclass[10pt,a4paper]{article}
\usepackage[utf8]{inputenc}
\usepackage{amsmath}
\usepackage{amsfonts}
\usepackage{amssymb}
\usepackage{makeidx}
\usepackage{notoccite}
\author{Wauyo Peter}
\title{Google Maps API}
\begin{document}
\maketitle

\begin{flushleft}
Google Maps is a web mapping service developed by Google\cite{1}. It offers satellite imagery, street maps, 360 degrees panoramic views of streets (Street View), real-time traffic conditions (Google Traffic), and route planning for traveling by foot, car, bicycle (in beta), or public transportation.\cite{2}
\end{flushleft}

\begin{flushleft}
An API is a set of methods and tools that can be used for building software applications.
Google Maps provides an API using which you can customize the maps and the information displayed on them.\cite{3}
\end{flushleft}

\begin{flushleft}
With the Google Maps Android API, you can add maps based on Google Maps data to your application. The API automatically handles access to Google Maps servers, data downloading, map display, and response to map gestures. You can also use API calls to add markers, polygons, and overlays to a basic map, and to change the user's view of a particular map area. These objects provide additional information for map locations, and allow user interaction with the map. The API allows you to add these graphics to a map:
\begin{itemize}
	\item Icons anchored to specific positions on the map (Markers).
	\item Sets of line segments (Polylines).
    \item Enclosed segments (Polygons).
    \item Bitmap graphics anchored to specific positions on the map (Ground Overlays).
    \item Sets of images which are displayed on top of the base map tiles (Tile Overlays).\cite{4}
\end{itemize}
\end{flushleft}

\begin{flushleft}
Google Maps APIs are categorized by platform: Web, Android and iOS.\cite{5}
\end{flushleft}

\bibliography{references}
\bibliographystyle{IEEEtrans}
\end{document}